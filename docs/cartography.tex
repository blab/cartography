% Template for PLoS
% Version 3.6 Aug 2022
%
% % % % % % % % % % % % % % % % % % % % % %
%
% -- IMPORTANT NOTE
%
% This template contains comments intended
% to minimize problems and delays during our production
% process. Please follow the template instructions
% whenever possible.
%
% % % % % % % % % % % % % % % % % % % % % % %
%
% Once your paper is accepted for publication,
% PLEASE REMOVE ALL TRACKED CHANGES in this file
% and leave only the final text of your manuscript.
% PLOS recommends the use of latexdiff to track changes during review, as this will help to maintain a clean tex file.
% Visit https://www.ctan.org/pkg/latexdiff?lang=en for info or contact us at latex@plos.org.
%
%
% There are no restrictions on package use within the LaTeX files except that no packages listed in the template may be deleted.
%
% Please do not include colors or graphics in the text.
%
% The manuscript LaTeX source should be contained within a single file (do not use \input, \externaldocument, or similar commands).
%
% % % % % % % % % % % % % % % % % % % % % % %
%
% -- FIGURES AND TABLES
%
% Please include tables/figure captions directly after the paragraph where they are first cited in the text.
%
% DO NOT INCLUDE GRAPHICS IN YOUR MANUSCRIPT
% - Figures should be uploaded separately from your manuscript file.
% - Figures generated using LaTeX should be extracted and removed from the PDF before submission.
% - Figures containing multiple panels/subfigures must be combined into one image file before submission.
% For figure citations, please use "Fig" instead of "Figure".
% See http://journals.plos.org/plosone/s/figures for PLOS figure guidelines.
%
% Tables should be cell-based and may not contain:
% - spacing/line breaks within cells to alter layout or alignment
% - do not nest tabular environments (no tabular environments within tabular environments)
% - no graphics or colored text (cell background color/shading OK)
% See http://journals.plos.org/plosone/s/tables for table guidelines.
%
% For tables that exceed the width of the text column, use the adjustwidth environment as illustrated in the example table in text below.
%
% % % % % % % % % % % % % % % % % % % % % % % %
%
% -- EQUATIONS, MATH SYMBOLS, SUBSCRIPTS, AND SUPERSCRIPTS
%
% IMPORTANT
% Below are a few tips to help format your equations and other special characters according to our specifications. For more tips to help reduce the possibility of formatting errors during conversion, please see our LaTeX guidelines at http://journals.plos.org/plosone/s/latex
%
% For inline equations, please be sure to include all portions of an equation in the math environment.  For example, x$^2$ is incorrect; this should be formatted as $x^2$ (or $\mathrm{x}^2$ if the romanized font is desired).
%
% Do not include text that is not math in the math environment. For example, CO2 should be written as CO\textsubscript{2} instead of CO$_2$.
%
% Please add line breaks to long display equations when possible in order to fit size of the column.
%
% For inline equations, please do not include punctuation (commas, etc) within the math environment unless this is part of the equation.
%
% When adding superscript or subscripts outside of brackets/braces, please group using {}.  For example, change "[U(D,E,\gamma)]^2" to "{[U(D,E,\gamma)]}^2".
%
% Do not use \cal for caligraphic font.  Instead, use \mathcal{}
%
% % % % % % % % % % % % % % % % % % % % % % % %
%
% Please contact latex@plos.org with any questions.
%
% % % % % % % % % % % % % % % % % % % % % % % %

\documentclass[10pt,letterpaper]{article}
\usepackage[top=0.85in,left=2.75in,footskip=0.75in]{geometry}

% amsmath and amssymb packages, useful for mathematical formulas and symbols
\usepackage{amsmath,amssymb}

% Use adjustwidth environment to exceed column width (see example table in text)
\usepackage{changepage}

% textcomp package and marvosym package for additional characters
\usepackage{textcomp,marvosym}

% cite package, to clean up citations in the main text. Do not remove.
\usepackage{cite}

% Use nameref to cite supporting information files (see Supporting Information section for more info)
\usepackage{nameref,hyperref}

% line numbers
\usepackage[right]{lineno}

% ligatures disabled
\usepackage[nopatch=eqnum]{microtype}
\DisableLigatures[f]{encoding = *, family = * }

% color can be used to apply background shading to table cells only
\usepackage[table]{xcolor}

% array package and thick rules for tables
\usepackage{array}

% create "+" rule type for thick vertical lines
\newcolumntype{+}{!{\vrule width 2pt}}

% create \thickcline for thick horizontal lines of variable length
\newlength\savedwidth
\newcommand\thickcline[1]{%
  \noalign{\global\savedwidth\arrayrulewidth\global\arrayrulewidth 2pt}%
  \cline{#1}%
  \noalign{\vskip\arrayrulewidth}%
  \noalign{\global\arrayrulewidth\savedwidth}%
}

% \thickhline command for thick horizontal lines that span the table
\newcommand\thickhline{\noalign{\global\savedwidth\arrayrulewidth\global\arrayrulewidth 2pt}%
\hline
\noalign{\global\arrayrulewidth\savedwidth}}

% TODO: replace with thickhline commands above?
\usepackage{booktabs}

% Inline comments for authors by initials.
\def\jhc#1{\textcolor{red}{[#1]}}
\def\snc#1{\textcolor{blue}{[#1]}}

% Remove comment for double spacing
\usepackage{setspace}
\doublespacing

% Text layout
\raggedright
\setlength{\parindent}{0.5cm}
\textwidth 5.25in
\textheight 8.75in

% Bold the 'Figure #' in the caption and separate it from the title/caption with a period
% Captions will be left justified
\usepackage[aboveskip=1pt,labelfont=bf,labelsep=period,justification=raggedright,singlelinecheck=off]{caption}
\renewcommand{\figurename}{Fig}

% Use the PLoS provided BiBTeX style
\bibliographystyle{plos2015}

% Remove brackets from numbering in List of References
\makeatletter
\renewcommand{\@biblabel}[1]{\quad#1.}
\makeatother



% Header and Footer with logo
\usepackage{lastpage,fancyhdr,graphicx}
\usepackage{epstopdf}
%\pagestyle{myheadings}
\pagestyle{fancy}
\fancyhf{}
%\setlength{\headheight}{27.023pt}
%\lhead{\includegraphics[width=2.0in]{PLOS-submission.eps}}
\rfoot{\thepage/\pageref{LastPage}}
\renewcommand{\headrulewidth}{0pt}
\renewcommand{\footrule}{\hrule height 2pt \vspace{2mm}}
\fancyheadoffset[L]{2.25in}
\fancyfootoffset[L]{2.25in}
\lfoot{\today}

%% Include all macros below

\newcommand{\lorem}{{\bf LOREM}}
\newcommand{\ipsum}{{\bf IPSUM}}

%% END MACROS SECTION


\begin{document}
\vspace*{0.2in}

% Title must be 250 characters or less.
\begin{flushleft}
{\Large
\textbf\newline{Genetic cartography reveals ancestral relationships of human pathogenic viruses} % Please use "sentence case" for title and headings (capitalize only the first word in a title (or heading), the first word in a subtitle (or subheading), and any proper nouns).
}
\newline
% Insert author names, affiliations and corresponding author email (do not include titles, positions, or degrees).
\\
Sravani Nanduri\textsuperscript{1},
Allison Black\textsuperscript{2},
Trevor Bedford\textsuperscript{2,3},
John Huddleston\textsuperscript{2*}
\\
\bigskip
\textbf{1} Paul G. Allen School of Computer Science and Engineering, University of Washington, Seattle, WA, USA
\\
\textbf{2} Vaccine and Infectious Disease Division, Fred Hutchinson Cancer Research Center, Seattle, WA, USA
\\
\textbf{3} Howard Hughes Medical Institute, Seattle, WA, USA
\\
\bigskip

% Insert additional author notes using the symbols described below. Insert symbol callouts after author names as necessary.
%
% Remove or comment out the author notes below if they aren't used.
%
% Primary Equal Contribution Note
%\Yinyang These authors contributed equally to this work.

% Additional Equal Contribution Note
% Also use this double-dagger symbol for special authorship notes, such as senior authorship.
%\ddag These authors also contributed equally to this work.

% Current address notes
% \textcurrency c Insert third current address

% Deceased author note
%\dag Deceased

% Group/Consortium Author Note
%\textpilcrow Membership list can be found in the Acknowledgments section.

% Use the asterisk to denote corresponding authorship and provide email address in note below.
* jhuddles@fredhutch.org

\end{flushleft}
% Please keep the abstract below 300 words
\section*{Abstract}
TBD.

% Please keep the Author Summary between 150 and 200 words
% Use first person. PLOS ONE authors please skip this step.
% Author Summary not valid for PLOS ONE submissions.
\section*{Author summary}
TBD.

\linenumbers

% Use "Eq" instead of "Equation" for equation citations.
\section*{Introduction}

Tracking the evolution of human pathogenic viruses in real time enables epidemiologists to respond quickly to emerging epidemics and local outbreaks \cite{Grubaugh2019}.
Real-time analyses of viral evolution typically rely on phylogenetic methods that can reconstruct the evolutionary history of viral populations from their genome sequences and estimate states of inferred ancestral viruses including their most likely genome sequence, time of circulation, and geographic location \cite{Volz2013,Baele2017,Sagulenko2018}.
Importantly, these methods assume that all sequence data share an evolutionary history represented by the clonal replication of genomes.
In practice, the evolutionary histories of many human pathogenic viruses violate this assumption through processes of reassortment or recombination, as seen in seasonal influenza \cite{Nelson2008,Marshall2013} and seasonal coronaviruses \cite{Su2016}, respectively.
Researchers account for these evolutionary mechanisms by limiting their analyses to individual genes \cite{Lemey2007,Bhatt2011}, combining multiple genes despite their different evolutionary histories \cite{Wiens1998}, or developing more sophisticated models to represent the joint likelihoods of multiple co-evolving lineages with ancestral reassortment or recombination graphs instead of phylogenetic trees \cite{Barrat-Charlaix2022,Muller2022}.
However, several key questions in genomic epidemiology do not require full phylogenetic inference of ancestral relationships and states.
For example, genomic epidemiologists commonly need to 1) identify clusters of closely-related genomes that represent regional outbreaks or new variants of concern \cite{OToole2022,McBroome2022,Stoddard2022}, 2) place newly sequenced viral genomes in the evolutionary context of other circulating strains \cite{OToole2021,Turakhia2021,Aksamentov2021}, and 3) visualize the genetic relationships among closely related virus samples \cite{Argimon2016,Campbell2021}.
Standard statistical methods can address these common use cases by transforming an alignment of genome sequences into a matrix and operating on that matrix to calculate pairwise relationships between sequences.
As these methods lack a formal biological model of evolutionary relationships, they make weak assumptions about the input data and therefore should be applicable to pathogen genomes that violate phylogenetic assumptions.

Common matrix representations of genome alignments include coding each distinct nucleotide character as an integer or calculating pairwise distances between sequences.
The first of these transformations is the first step prior to performing a principal components analysis (PCA) to find orthogonal representations of the inputs that explain the most variance \cite{jolliffe_cadima_2016}.
The second transformation calculates the number of mismatches between each pair of aligned genome sequences, also known as the Hamming distance, to create a distance matrix.
Most phylogenetic methods begin by building a distance matrix for all sequences in a given multiple sequence alignment.
Dimensionality reduction algorithms such as multidimensional scaling (MDS) \cite{hout_papesh_goldinger_2012}, t-SNE \cite{maaten2008visualizing}, and UMAP \cite{lel2018umap} accept such distance matrices as an input and produce a corresponding low-dimensional representation or ``embedding'' of those data.
Both types of transformation allow us to reduce high-dimensional genome alignments ($M \times N$ values for $M$ genomes of length $N$) to low-dimensional embeddings where clustering algorithms and visualization are more tractable.
Additionally, distance-based methods can reflect the presence or absence of insertions and deletions in an alignment that phylogenetic methods ignore.

Each of the embedding methods mentioned above has been applied previously to genomic data to identify clusters of related genomes and visualize relationships between individuals.
Although PCA is a generic linear algebra algorithm that optimizes for an orthogonal embedding of the data, the principal components from single nucleotide polymorphisms (SNPs) represent mean coalescent times and therefore recapitulate broad phylogenetic relationships \cite{mcvean_2009}.
PCA has been applied to SNPs of human genomes \cite{novembre_2008,alexander_2009,mcvean_2009,auton_2015} and to multiple sequence alignments of viral genomes \cite{metsky_2017}.
MDS attempts to embed input data into a lower-dimensional representation such that each pair of data points are as far apart in the embedding as they are in the original data.
MDS has been applied to multiple gene segments of seasonal influenza viruses to visualize evolutionary relationships between segments \cite{rambaut_2008}.
Both t-SNE and UMAP build on manifold learning methods like MDS to find low-dimensional embeddings of data that place similar points close together and dissimilar points far apart \cite{kobak_2021}.
These methods have been applied to SNPs from human genomes \cite{diaz-papkovich_2019} and single-cell transcriptomes \cite{becht_2018,kobak_2019}.

Although these methods are commonly used for qualitative studies of evolutionary relationships, few studies have attempted to quantify patterns observed in the resulting embeddings and no studies have investigated the value of applying these methods to human pathogenic viruses.
To this end, we tuned and validated the performance of PCA, MDS, t-SNE, and UMAP with genomes from simulated influenza-like and coronavirus-like populations and then applied these methods to natural populations of seasonal influenza virus A/H3N2 and SARS-CoV-2.
These natural viruses are highly relevant as major causes of global human mortality, common subjects of real-time genomic epidemiology, and representatives of reassortant and recombinant human pathogens.
For each combination of virus and embedding method, we quantified the relationship between pairwise genetic and Euclidean embedding distances, identified clusters of closely-related genomes in embedding space, and evaluated the accuracy of clusters compared to genetic groups defined by experts and biologically-informed models.
Finally, we tested the ability of these methods to identify reassortment of seasonal influenza virus hemagglutinin (HA) and neuraminidase (NA) segments and recombination in SARS-CoV-2 genomes.
These results inform our recommendations for future applications of these methods including which are most effective for specific problems in genomic epidemiology and which parameters researchers should use for each method.

\section*{Materials and methods}

\subsection*{Embedding methods}

We selected four standard and common dimensionality reduction (or ``embedding'') methods to apply to human pathogenic viruses: PCA, MDS, t-SNE, and UMAP.
PCA operates on a matrix with samples in rows, ``features'' in columns, and numeric values in each cell \cite{jolliffe_cadima_2016}.
To apply PCA to multiple sequence alignments, we transformed each nucleotide value into a corresponding integer (A to 1, G to 2, C to 3, T to 4, and all other values to 5) and applied scikit-learn's PCA implementation to the resulting numerical matrix with the ``full'' singular value decomposition solver and 10 components \cite{Pedregosa2011}.

The remaining three methods operate on a distance matrix.
We constructed a distance matrix from a multiple sequence alignment by calculating the pairwise Hamming distance between nucleotide sequences.
By default, the Hamming distance only counted mismatches between pairs of standard nucleotide values (A, C, G, and T), ignoring other values including gaps.
We implemented an optional mode that additionally counted each occurrence of consecutive gap characters in either input sequence as individual insertion/deletion (``indel'') events.

We applied scikit-learn's MDS implementation to a given distance matrix, with an option to set the number of components in the resulting embedding \cite{Pedregosa2011}.
Similarly, we applied scikit-learn's t-SNE implementation, with options to set the ``perplexity'' and the ``learning rate''.
The perplexity controls the number of neighbors the algorithm uses per input sample to determine an optimal embedding \cite{maaten2008visualizing}.
This parameter effectively determines the balance between maintaining ``local'' or ``global'' structure in the embedding \cite{kobak_2019}.
The learning rate controls how rapidly the t-SNE algorithm converges on a specific embedding \cite{Jacobs1988,maaten2008visualizing} and should scale with the number of input samples \cite{Belkina2019}.
We initialized t-SNE embeddings with the first two components of the corresponding PCA embedding, as previously recommended to obtain more accurate global structure \cite{kobak_2019,kobak_2021}.
Finally, we applied the \textit{umap-learn} Python package written by UMAP's authors, with options to set the number of ``nearest neighbors'' and the ``minimum distance'' \cite{lel2018umap}.
As with t-SNE's perplexity parameter, the nearest neighbors parameter determines how many adjacent samples the UMAP algorithm considers per sample to find an optimal embedding.
The minimum distance sets the lower limit for how close any two samples can map next to each other in a UMAP embedding.
Lower minimum distances allow tighter groups of samples to form.
For both t-SNE and UMAP, we used the default number of components of 2.

\subsection*{Simulation of influenza-like and coronavirus-like populations}

Given the relative lack of prior application of dimensionality reduction methods to human pathogenic viruses, we first attempted to understand the behavior and optimal parameter values for these methods when applied to simulated viral populations with well-defined evolutionary parameters.
To this end, we simulated populations of influenza-like and coronavirus-like viruses using SANTA-SIM \cite{Jariani2019}.
These simulated populations allowed us to identify optimal parameters for each embedding method, without overfitting to the limited data available for natural viral populations.
For each population type described below, we simulated five independent replicates with fixed random seeds for over 55 years, filtered out the first 10 years of each population as a burn-in period, and analyzed the remaining years.

We simulated influenza-like populations as previously described with 1,700 bp hemagglutinin sequences \cite{Huddleston2020}.
As in that previous study, we scaled the number of simulated generations per real year to 200 per year to match the observed mutation rate for natural H3N2 HA sequences, and we sampled 10 genomes every 4 generations for 12,000 generations (or 60 years of real time).

We simulated coronavirus-like populations as previously described for human seasonal coronaviruses with genomes of 21,285 bp \cite{Muller2022}.
For the current study, we assigned 30 generations per real year to obtain mutation rates similar to the $8 \times 10^{-4}$ substitutions per site per year estimated for SARS-CoV-2 \cite{Rambaut2020}.
To account for the effect of recombination on optimal method parameters, we simulated populations with a recombination rate of $10^{-5}$ events per site per year based on human seasonal coronaviruses for which recombination rates are well-studied \cite{Muller2022,Carabelli2023}.
We calibrated the overall recombination probability in SANTA-SIM such that the number of observed recombination events per year matched the expected number for human seasonal coronaviruses (0.3 per year) \cite{Muller2022}.
To assist with this calibration of recombination events per year, we modified the SANTA-SIM source code to emit a boolean status of ``is recombinant'' for each sampled genome.
This change allowed us to identify recombinant genomes by their metadata in downstream analyses and calculate the number of recombination events observed per year.
For each replicate population, we sampled 15 genomes every generation for 1,700 generations (or approximately 56 years of real time).

\subsection*{Optimization of embedding method parameters}

We identified optimal parameter values for each embedding method with time series cross-validation of embeddings based on simulated populations \cite{HyndmanAthanasopoulos2021}.
To increase the interpretability of embedding space, we defined parameters as ``optimal'' when they maximized the linear relationship between pairwise genetic distance of viral genomes and the corresponding Euclidean distance between those same genomes in an embedding.
This optimization approach allowed us to also determine the degree to which each method could recapitulate this linear relationship.

For each simulated population replicate, we created 10 training and test datasets that each consisted of 4 years of training data and 4 years of test data preceded by a 1-year gap from the end of the training time period.
These settings produced training/test data with 2000 samples each for influenza-like populations and 1800 samples each for coronavirus-like populations.
For each combination of training/test dataset, embedding method, and method parameters, we applied the following steps.
We created an embedding from the training data with the given parameters, fit a linear model to estimate pairwise genetic distance from pairwise Euclidean distance in the embedding, created an embedding from the test data, estimated the pairwise genetic distance for genomes in the test data based on their Euclidean distances and the linear model fit to the training data, and calculated the mean absolute error (MAE) between estimated and observed genetic distances in the test data.
We summarized the error for a given population type, method, and method parameters across all population replicates and training/test data by calculating the median of the MAE.
For all method parameters except those controlling the number of components used for the embedding, we selected the optimal parameters as those that minimized the median MAE for a given embedding method.
Since increasing the number of components used by PCA and MDS allows these methods to overfit to available data, we selected the optimal number of components for these methods as the number beyond which the median MAE did not decrease by at least 1 nucleotide.
This approach follows the same concept from the MDS algorithm itself where optimization occurs iteratively until some predefined error threshold is reached.

With the approach described above, we tested each method across a range of relevant parameters with all combinations of parameter values.
For PCA, we tested the number of components between 2 and 6.
For MDS, we tested the number of components between 2 and 10.
\jhc{The difference in number of components between PCA and MDS sticks out here. We should use the same number for both or justify using different numbers.}
For t-SNE, we tested perplexity values of 15, 30, 100, 200, and 300, and we tested learning rates of 100, 200, and 500.
For UMAP, we tested nearest neighbor values of 25, 50, and 100, and we tested values for the minimum distance that points can be in an embedding of 0.05, 0.1, and 0.25.

\subsection*{Selection of natural virus population data}

We selected recent publicly available genome sequences and metadata for seasonal influenza A/H3N2 HA and NA genes and SARS-CoV-2 genomes from INSDC databases \cite{Arita2021}.
For both viruses, we divided the available data into an ``early'' and ``late'' dataset to use as training and test data, respectively, for identification of virus-specific clustering parameters.

For analyses that focused only on A/H3N2 HA data, we defined the early dataset between October 2015 and April 2018 and the late dataset between April 2018 to January 2020.
For both early and late datasets, we evenly sampled 25 sequences per country, year, and month, excluding known outliers.
With this sampling scheme, we selected 1,918 HA sequences for the early dataset and 821 for the late dataset.
For analyses that combined A/H3N2 HA and NA data, we defined a single dataset between January 2016 and July 2018, keeping 1,643 samples for which both HA and NA have been sequenced.

For SARS-CoV-2 data, we defined the early dataset between January 1, 2020 and January 1, 2022 and the late dataset between January 1, 2022 and July 5, 2023.
For the early dataset, we evenly sampled 1,734 SARS-CoV-2 genomes by geographic region, year, and month, excluding known outliers.
For the late dataset, we used the same even sampling by space and time to select 1,394 representative genomes.
In addition to these genomes, we sampled at most 20 genomes per Nextclade pango lineage for 10 known recombinant lineages (XAY, XBB, XBB.1, XBC, XBF, XBL, XC, XD, XE, XF, and XG) and their corresponding parental lineages (AY.29, AY.4, AY.45, B.1.1.7, B.1.617, BA.1, BA.2, BA.2.75, BA.4, BA.5, BA.5.2.3, BJ.1, BM.1.1.1, and CJ.1) as defined by \href{https://libguides.mskcc.org/SARS2/recombination}{https://libguides.mskcc.org/SARS2/recombination}.
With these additional genomes, the late SARS-CoV-2 dataset included 2,072 total genomes.

\subsection*{Evaluation of linear relationships between genetic distance and Euclidean distance in embeddings}

To evaluate the biological interpretability of distances between samples in low-dimensional embeddings, we plotted the pairwise Euclidean distance between samples in each embedding against the corresponding genetic distance between the same samples.
We calculated Euclidean distance using all components of the given embedding (e.g., 2 components for PCA, t-SNE, and UMAP and 3 components for MDS).
For each embedding, we fit a linear model between Euclidean and genetic distance and calculated the squared Pearson's correlation coefficient, $R^{2}$.
The distance plots provide a qualitative assessment of each embedding's local and global structure relative to a biologically meaningful scale of genetic distance, while the linear models and correlation coefficients quantify the global structure in the embeddings.

\subsection*{Phylogenetic analysis}

For each natural population described above, we created an annotated, time-scaled phylogenetic tree.
For seasonal influenza A/H3N2 HA and NA sequences, we aligned sequences with MAAFT (version 7.486) \cite{Katoh2002,Katoh2013} using the \textit{augur align} command (version 22.0.3) \cite{Huddleston2021}.
For SARS-CoV-2 sequences, we used existing reference-based alignments provided by the Nextstrain team (\href{https://docs.nextstrain.org/projects/ncov/en/latest/reference/remote_inputs.html#summary-of-available-genbank-open-files}{https://docs.nextstrain.org/projects/ncov/en/latest/reference/remote\_inputs.html}) and generated with Nextalign (version 2.14.0) \cite{Aksamentov2021}.
We inferred a phylogeny with IQ-TREE (version 2.1.4-beta) \cite{Nguyen2014} using the \textit{augur tree} command and inferred a time tree with TreeTime (version 0.8.2) \cite{Sagulenko2018} using the \textit{augur refine} command.
We visualized phylogenies with Auspice \cite{Hadfield2018}, after first converting the trees to Auspice JSON format with \textit{augur export}.

\subsection*{Definitions of genetic groups by experts or biologically-informed models}

We annotated phylogenetic trees with genetic groups previously identified by experts or assigned by biologically-informed models.
For seasonal influenza A/H3N2, the World Health Organization assigns ``clade'' labels to clades in HA phylogenies that appear to be genetically or phenotypically distinct from other recently circulating A/H3N2 samples.
We used the latest clade definitions for A/H3N2 maintained by the Nextstrain team as part of their seasonal influenza surveillance efforts \cite{Neher2015}.

As seasonal influenza clades only account for the HA gene and lack information about reassortment events, we assigned joint HA and NA genetic groups using a biologically-informed model, TreeKnit \cite{Barrat-Charlaix2022}.
TreeKnit infers ancestral reassortment graphs from two gene trees, finding groups of samples for which both genes share the same history.
These groups, also known as maximally compatible clades (MCCs), represent samples whose HA and NA genes have reassorted together.
We applied TreeKnit to HA and NA trees produced by IQ-TREE, as described above, using a gamma value of 2.5.
TreeKnit identified 259 MCCs for 1,286 samples with an average of 6 samples per MCC.
We filtered these MCCs to keep only those with at least 10 samples, resulting in 28 MCCs that represented 1,286 samples (78\%) with an average of 46 samples per MCC.

For SARS-CoV-2, we used both expert-defined ``Nextstrain clades'' \cite{Hodcroft2020,Bedford2021,Roemer2022} and computationally-defined Pangolin lineages \cite{OToole2021} provided by the Nextclade pango implementation as ``Nextclade pango'' annotations.
Nextstrain clade definitions represent the World Health Organization's variants of concern and other phylogenetic clades that have reached minimum global and regional frequencies and growth rates.
Pangolin lineages represent a combination of lineages assigned by a machine learning model (\href{https://cov-lineages.org/resources/pangolin/pangolearn.html}{pangoLEARN}) and expert-curated lineages (\href{https://github.com/cov-lineages/pango-designation}{https://github.com/cov-lineages/pango-designation}) and must contain at least 5 samples with an unambiguous evolutionary event.
As such, Nextstrain clades represent a much coarser genetic resolution than Pangolin lineages.
Additionally, Pangolin lineages produced by recombination receive a lineage name prefixed by an ``X'', while Nextstrain clades do not explicitly reflect recombination events.

Since Pangolin lineages can represent much smaller genetic groups than are practically useful, we collapsed Pango lineages with fewer than 10 samples in our analysis into their parental lineages using the pango\_aliasor tool (\href{https://github.com/corneliusroemer/pango_aliasor}{https://github.com/corneliusroemer/pango\_aliasor}).
Specifically, we counted the number of samples per lineage, sorted lineages in ascending order by count, and collapsed each lineage with a count less than 10 into its parental lineage in the count-sorted order.
This approach allowed small lineages to aggregate with other small parental lineages and meet the 10-sample threshold.
We used these ``Nextclade pango collapsed'' lineages for subsequent analyses.

\subsection*{Clustering of samples in embeddings}

To understand how well embeddings of genetic data could capture previously defined genetic groups, we applied an unsupervised clustering algorithm, HDBSCAN \cite{campello2015hierarchical}, to each embedding.
HDBSCAN identifies initial clusters from high-density regions in the input space and merges these clusters hierarchically.
This algorithm allowed us to avoid defining an arbitrary or biased expected number of clusters \emph{a priori}.
HDBSCAN provides parameters to tune the minimum number of samples required to seed an initial cluster (``min samples''), the minimum size for a final cluster (``min size''), and the minimum distance between initial clusters below which those clusters are hierarchically merged (``distance threshold'').
We hardcoded the min samples to 5 to minimize the number of spurious initial clusters and min size to 10 to reflect our interest in genetic groups with at least 10 samples throughout our analyses.
HDBSCAN calculates the distance between clusters on the Euclidean scale of each embedding.
To account for embedding-specific distances, we performed a coarse grid search of distance threshold values for each virus type and embedding method.

We performed the grid search on the early datasets for both seasonal influenza A/H3N2 HA and SARS-CoV-2.
For each dataset and embedding method, we applied HDBSCAN clustering with a distance threshold between 0 and 7 inclusive with steps of 0.5 between values.
For a given threshold, we obtained sets of samples assigned to HDBSCAN clusters from the embedding.
We evaluated the accuracy of these clusters with variation of information (VI) which calculates the distance between two sets of clusters of the same samples \cite{meilua2003comparing}.
When two sets of clusters are identical, VI equals 0.
When the sets are maximally different, VI is $\log{N}$ where $N$ is the total number of samples.
To make VI values comparable across datasets, we normalized each value by dividing by $\log{N}$, following the pattern used to validate TreeKnit's MCCs \cite{Barrat-Charlaix2022}.
For each virus dataset and embedding method, we identified the distance threshold that minimized the normalized VI between HDBSCAN clusters and genetic groups defined by experts or biologically-informed models (``Nextstrain clade'' for seasonal influenza and both ``Nextstrain clade'' and ``collapsed Nextclade pango lineage'' for SARS-CoV-2).
HDBSCAN allows samples to not belong to a cluster and assigns these samples a numeric label of -1.
We intentionally included all unassigned samples in the normalized VI calculation thereby penalizing cluster parameters that increased the number of unassigned samples by increasing their VI values.
Finally, we used these optimal distance thresholds to identify clusters in out-of-sample data from the late datasets for both viruses and calculate the normalized VI between those clusters and previously defined genetic groups.

\subsection*{Identification of cluster-specific mutations}

To better understand the genetic basis of embedding clusters, we identified cluster-specific mutations for all HDBSCAN clusters.
First, we found all mutations between each sample's sequence and the reference sequence used to produce the alignment, considering only A, C, G, T, and gap characters.
Within each cluster, we identified mutations that occurred in at least 10 samples and in at least 50\% of samples in the cluster.
We recorded the resulting mutations per cluster in a table with columns for the embedding method, the position of the mutation, the derived allele of the mutation, and a list of the distinct clusters the mutation appeared in.
From this table, we could identify mutations the only occurred in specific clusters and mutations that distinguished sets of clusters from each other.

\subsection*{Assessment of HA/NA reassortment in seasonal influenza populations}

To assess the ability of embedding methods to detect reassortment in seasonal influenza populations, we applied each method to either HA alignments only or concatenated alignments of HA and NA sequences from the same samples, performed HDBSCAN clustering with the optimal distance threshold for the given method, and calculated the normalized VI between the resulting clusters and TreeKnit MCCs.
We compared normalized VI values for the HA-only clusters of each method to the corresponding VI values for the HA/NA clusters.
Lower VI values in the HA/NA clusters than HA-only clusters indicated better clustering of samples into known reassortment groups.

\subsection*{Assessment of recombination in SARS-CoV-2 populations}

To assess the ability of embedding methods to detect recombination in late SARS-CoV-2 populations (2022-2023), we calculated the Euclidean distances in low-dimensional space between the 10 known recombinant lineages and their respective parental lineages described in ``Selection of natural virus population data'' above.
Given that we optimized each method's parameters to maximize a linear relationship between genetic and Euclidean distance, we expected embeddings to place recombinant lineages between their parental lineages, reflecting the intermediate genetic state of the recombinants.
For a recombinant lineage $X$ and its parental lineages $A$ and $B$, we calculated the average pairwise Euclidean distance, $D$, between samples in $A$ and $B$, $A$ and $X$, and $B$ and $X$.
We identified lineages that mapped properly as those for which $D(A, X) < D(A, B)$ and $D(B, X) < D(A, B)$.
We also identified lineages for which the recombinant lineage placed closer to at least one parent than the distance between the parents.
Note that we used the original uncollapsed ``Nextclade pango'' annotations to identify samples in each lineage, as these were the lineage names used to include recombinant samples in the analysis and define known relationships between recombinant and parental lineages.

\subsection*{Data and software availability}

The entire workflow for our analyses was implemented with Snakemake \cite{molder_2021}.
We have provided all source code, configuration files, and datasets at \href{https://github.com/blab/cartography}{https://github.com/blab/cartography}.
Interactive phylogenetic trees and corresponding embeddings for natural populations are available at \href{https://nextstrain.org/community/blab/cartography/}{https://nextstrain.org/community/blab/cartography/}.
The \textit{pathogen-embed} Python package, available on PyPI, provides command line utilities to calculate distance matrices (\textit{pathogen-distance}), calculate embeddings per method (\textit{pathogen-embed}), and apply hierarchical clustering to embeddings (\textit{pathogen-cluster}).

% Results and Discussion can be combined.
\section*{Results}

\subsection*{Simulated populations enable tuning of embedding method parameters}

To understand how well PCA, MDS, t-SNE, and UMAP could represent genetic relationships between human pathogen viruses under well-defined evolutionary conditions, we simulated influenza-like and coronavirus-like populations, created embeddings for each population across a range of method parameters, and identified optimal parameters as those that maximized a linear relationship between genetic distance and Euclidean distance in low-dimensional space (see Methods).
Specifically, we selected parameters that minimized the median of the mean absolute error (MAE) between observed pairwise genetic distances of simulated genomes and predicted genetic distances for those genomes based on their Euclidean distances in each embedding.
For methods like PCA and MDS where increasing the number of components available to the embedding could lead to overfitting, we selected the maximum number of components beyond which the median MAE did not decrease by more than 1 nucleotide.

For influenza-like populations, the optimal parameters were 2 components for PCA, 3 components for MDS, perplexity of 100 and learning rate of 200 for t-SNE, and nearest neighbors of 100 and minimum distance of 0.25 for UMAP.
As expected, increasing the number of components for PCA and MDS gradually decreased the median MAEs of their embeddings (\nameref{S1_Fig_simulated_flu_errors} A and B).
However, beyond 2 and 3 components, respectively, the reduction in error did not exceed 1 nucleotide.
This result suggests that there were diminishing returns for the increased complexity of additional components.
Both t-SNE and UMAP embeddings produced a wide range of errors (the majority between 10 and 20 average mismatches) across all parameter values (\nameref{S1_Fig_simulated_flu_errors} C and D).
Embeddings from t-SNE appeared robust to variation in parameters, with a slight improvement in median MAE associated with perplexity of 100 and little benefit to any of the learning rate values (\nameref{S1_Fig_simulated_flu_errors} C).
\jhc{Based on these results, we should consider setting the learning rate to the default for scikit-learn which scales the rate with the input sample size.}
Similarly, UMAP embeddings were robust across the range of tested parameters, with the greatest benefit coming from setting the nearest neighbors greater than 25 and no benefit from changing the minimum distance between points (\nameref{S1_Fig_simulated_flu_errors} D).

% TODO: remove supporting information figures in final submission; figures must be uploaded separately from the manuscript.
\begin{figure}[!h]
  \includegraphics[width=\columnwidth]{figures/simulated-influenza-like-with-no-reassortment-scores-by-parameters.png}
  \caption*{{\bf S1 Fig. Distribution of mean absolute errors (MAE) between observed and predicted pairwise genetic distances per embedding method parameters for simulated influenza-like populations.}}
\end{figure}

The optimal parameters for coronavirus-like populations were nearly the same as those for the influenza-like populations.
The optimal parameters were 2 components for PCA, 3 for MDS, perplexity of 100 and learning rate of 500 for t-SNE, and nearest neighbors of 100 and minimum distance of 0.05 for UMAP.
As with influenza-like populations, both PCA and MDS showed diminishing benefits of increasing the number of components (\nameref{S2_Fig_simulated_coronavirus_errors} A and B).
Similarly, we observed little improvement in MAEs from varying t-SNE and UMAP parameters (\nameref{S2_Fig_simulated_coronavirus_errors} C and D).
The most noticeable improvement came from setting t-SNE's perplexity to 100 (\nameref{S2_Fig_simulated_coronavirus_errors} C).
These results indicate the limits of t-SNE and UMAP to represent global genetic structure, at least across the parameter regimes considered here.
\jhc{An obvious follow-up question would be whether we can improve MAEs for these methods by increasing components available to them, too.}

% TODO: remove supporting information figures in final submission; figures must be uploaded separately from the manuscript.
\begin{figure}[!h]
\includegraphics[width=\columnwidth]{figures/simulated-coronavirus-like-with-moderate-recombination-rate-scores-by-parameters.png}
\caption*{{\bf S2 Fig. Distribution of mean absolute errors (MAE) between observed and predicted pairwise genetic distances per embedding method parameters for simulated coronavirus-like populations.}}
\end{figure}

We inspected representative embeddings based on the optimal parameters above for the first four years of influenza- and coronavirus-like populations.
Simulated sequences collected from the same time period tended to map closer in embedding space, indicating the maintenance of ``local'' genetic structure in the embeddings (Fig.~\ref{fig:simulated-populations-representative-embeddings}).
Most embeddings also represented some form of global structure, with later generations mapping closer to intermediate generations than earlier generations.
MDS maintained the greatest continuity between generations for both population types (\nameref{S3_Fig_simulated_representative_mds_embeddings}).
In contrast, PCA, t-SNE, and UMAP all demonstrated tighter clusters of samples separated by potentially arbitrary space.
The UMAP embedding for the coronavirus-like samples was most extreme in this respect, with a tight cluster of early samples placing far away from all other samples in the embedding including those from nearby generations.
These qualitative results matched our expectations based on how well each method maximized a linear relationship between genetic and Euclidean distances during parameter optimization (\nameref{S1_Fig_simulated_flu_errors} and \nameref{S2_Fig_simulated_coronavirus_errors}).

\begin{figure}[!h]
% TODO: remove includegraphics commands in final submission; figures must be uploaded separately from the manuscript.
\includegraphics[width=\columnwidth]{figures/simulated-populations-representative-embeddings.png}
\caption{{\bf Representative embeddings for simulated populations using optimal parameters per pathogen (rows) and embedding method (columns).}
  Each panel represents an embedding method for a single replicate population for the corresponding pathogen type.
  Each point represents a simulated viral sequence colored by its generation with darker values representing later generations.
  The MDS embedding shows the first two of three total dimensions.
  \nameref{S3_Fig_simulated_representative_mds_embeddings} shows the full MDS embedding for all components.}
\label{fig:simulated-populations-representative-embeddings}
\end{figure}

\begin{figure}[!h]
% TODO: remove includegraphics commands in final submission; figures must be uploaded separately from the manuscript.
\includegraphics[width=\columnwidth]{figures/simulated-populations-representative-mds-embeddings.png}
\caption*{{\bf S3 Fig. Representative MDS embeddings for simulated populations using optimal parameters per pathogen (rows) and showing all three components.}}
\end{figure}

\subsection*{Embedding clusters recapitulate phylogenetic clades for seasonal influenza A/H3N2}

Seasonal influenza A/H3N2's hemagglutinin (HA) sequences provide an ideal positive control to test embedding methods and clustering in low-dimensional space.
A/H3N2's HA protein evolves rapidly, accumulating amino acid mutations that enable escape from adaptive immunity in human populations \cite{Petrova2018}.
These mutations produce distinct phylogenetic clades that represent potentially different antigenic phenotypes.
The World Health Organization (WHO) Global Influenza Surveillance and Response System (GISRS) regularly sequences genomes of circulating influenza lineages \cite{Hay2018} and submits these sequences to public INSDC databases like NCBI's GenBank \cite{Arita2021}.
These factors, coupled with HA's relatively short gene size of 1,701 nucleotides, facilitate real-time genomic epidemiology of A/H3N2 \cite{Neher2015} and rapid analysis by the embedding methods we wanted to evaluate.

We first applied each embedding method to the ``early'' H3N2 data collected from 2016 through 2018, colored samples by previously defined phylogenetic clades, and inspected the placement of these samples in the embeddings and corresponding phylogeny.
All four embedding methods qualitatively recapitulated clade-level groupings observed in the phylogeny (Fig~\ref{fig:seasonal-influenza-h3n2-ha-embeddings}).
Samples from the same clade generally grouped tightly together.
Most embedding methods also clearly delineated larger phylogenetic clades, placing clades A1, A2, A3, A4, and 3c3.A into separate locations in the embeddings.
One exception to this pattern was the PCA embedding which grouped samples from clades A3 and A4 into the same space.
Despite maintaining local and broader global structure, not all embeddings captured intermediate genetic structure.
For example, clade A1b descended from clade A1 and diversified into the smaller subclades A1b/131K, A1b/135K, and A1b/135N.
All methods placed A1b far from its ancestor A1, but PCA, t-SNE, and UMAP all placed descendants of A1b into tight clusters together.
MDS was the only method that clearly separated the descendants of A1b into their own clusters.
The t-SNE embedding also created separate clusters of the three descendants, but these clusters all placed so close together in the embedding space that, without previously defined clade labels, we would have visually grouped these samples into a single cluster.
These results qualitatively replicate the patterns we observed in embeddings for simulated influenza-like populations (Fig~\ref{fig:simulated-populations-representative-embeddings}).

% For figure citations, please use "Fig" instead of "Figure".

% Place figure captions after the first paragraph in which they are cited.
\begin{figure}[!h]
% TODO: remove includegraphics commands in final submission; figures must be uploaded separately from the manuscript.
\includegraphics[width=\columnwidth]{figures/flu-2016-2018-ha-embeddings-by-clade.png}
\caption{{\bf The phylogeny of influenza A/H3N2 viruses (top) shows the evolutionary relationships among viruses including clades, or viruses that share the same mutations and descend from the same common ancestor.}
Reduced dimensionality embeddings of genetic sequences into two dimensions by PCA (middle left), MDS (middle right), t-SNE (bottom left), and UMAP (bottom right) generally recapitulate groups of viruses into clades without inferring ancestral relationships.}
\label{fig:seasonal-influenza-h3n2-ha-embeddings}
\end{figure}

To quantify the apparent maintenance of local and global structure by all four embedding methods, we calculated the relationship between pairwise genetic and Euclidean distance of samples in each embedding.
All four methods maintained a linear relationship between genetic and Euclidean distances for samples that differed by no more than $\approx$10 nucleotides (Fig~\ref{fig:seasonal-influenza-h3n2-ha-pairwise-distances}).
However, only MDS consistently maintained that linearity as genetic distance increased (Pearson's $R^{2} = 0.942$).
Values of Euclidean distances in MDS corresponded nearly perfectly with values of genetic distances.
In contrast, we observed a nonlinear relationship for samples with more genetic differences in PCA (Pearson's $R^{2} = 0.689$), t-SNE (Pearson's $R^{2} = 0.502$), and UMAP (Pearson's $R^{2} = 0.447$) embeddings.
Although the most genetically distant samples mapped far from each other in all of these embeddings, samples with intermediate distances could map much closer or farther than expected by a linear model.
In t-SNE and UMAP embeddings, some pairs of samples with intermediate distances of 30-40 nucleotides mapped farther apart than pairs of samples with much greater distances.

\begin{figure}[!h]
\includegraphics[width=\columnwidth]{figures/flu-2016-2018-ha-euclidean-distance-by-genetic-distance.png}
\caption{{\bf Relationship between pairwise genetic distance and Euclidean distance in embedding space, representing the ability of embedding methods to capture local and global genetic structure.}
Embeddings with PCA (upper left), MDS (upper right), t-SNE (lower left), and UMAP (lower right) consistently exhibit linear relationships for pairs of strains that differ by no more than 10 nucleotides.}
\label{fig:seasonal-influenza-h3n2-ha-pairwise-distances}
\end{figure}

Next, we measured how well clusters of H3N2 HA samples in each embedding corresponded to previously defined genetic groups.
For each embedding, we assigned cluster labels to each sample with the hierarchical clustering algorithm, HDBSCAN, which does not require an expected number of clusters as input \cite{campello2015hierarchical}.
HDBSCAN does require definition of a minimum distance that its initial clusters must be from each other to avoid being merged into the same cluster.
This distance corresponds to the Euclidean distance between clusters in embedding space which varies by method (Fig~\ref{fig:seasonal-influenza-h3n2-ha-pairwise-distances}).
To find the optimal minimum distance for HDBSCAN clusters of H3N2 HA data, we assigned clusters to each embedding for a range of distance values (0-7) with a step size of 0.5 and calculated the accuracy of clusters at each distance value compared to the Nextstrain clade assignments shown in Fig~\ref{fig:seasonal-influenza-h3n2-ha-embeddings}.
We selected the minimum distance value per method that minimized the difference between HDBSCAN clusters and clade assignments as measured by the normalized variation of information (VI) metric \cite{meilua2003comparing} (see Methods).
The optimal minimum distances were 0.5 for PCA, 3.5 for MDS, 2.5 for t-SNE, and 1.5 for UMAP (Table~\ref{table:accuracy}).
Since Euclidean distances for MDS correspond directly to genetic distances, these results show that clusters must be at least 3.5 nucleotides apart to be considered distinct.

% Place tables after the first paragraph in which they are cited.
\begin{table}[!ht]
%\begin{adjustwidth}{-2.25in}{0in} % Comment out/remove adjustwidth environment if table fits in text column.
\centering
\caption{
  {\bf Accuracy of embedding methods per human pathogenic virus sorted by normalized variation of information (VI) distance.}
  Smaller VI values indicate smaller distances between HDBSCAN clusters and known genetic groups with 0 indicating identical clusters and 1 indicating maximally different clusters.
  Threshold refers to the distance threshold used to assign clusters with HDBSCAN.}
\begin{tabular}{llrr}
\toprule
                     Pathogen & Method &   VI &  Threshold \\
\midrule
               Influenza H3N2 &  t-SNE & 0.04 &        2.0 \\
                              &   UMAP & 0.08 &        1.0 \\
                              &    MDS & 0.10 &        3.5 \\
                              &    PCA & 0.19 &        0.5 \\
SARS-CoV-2 (Nextstrain clade) &  t-SNE & 0.07 &        1.0 \\
                              &    MDS & 0.15 &        0.0 \\
                              &   UMAP & 0.16 &        0.5 \\
                              &    PCA & 0.22 &        0.5 \\
           SARS-CoV-2 (Pango) &  t-SNE & 0.12 &        1.0 \\
                              &    MDS & 0.23 &        0.0 \\
                              &   UMAP & 0.25 &        0.5 \\
                              &    PCA & 0.31 &        0.5 \\
\bottomrule
\end{tabular}

\label{table:accuracy}
%\end{adjustwidth}
\end{table}

As expected, the clusters for each method generally corresponded to larger phylogenetic clades (Fig~\ref{fig:seasonal-influenza-h3n2-ha-2016-2018-clusters}).
Clusters from UMAP most accurately captured expert clade assignments (normalized VI=0.07) with 6 clusters.
These clusters captured broader phylogenetic clades (A1, A1b, A2, A3, 3c2.A, and 3c3.A) but failed to distinguish between smaller or less divergent clades (A4 and A1b's descendants).
Clusters from t-SNE performed nearly as well (normalized VI=0.08) with 11 clusters.
These clusters also captured broader clades (including the A4 clade that UMAP clustered with 3c2.A) and failed to distinguish among A1b and its subclades.
Interestingly, t-SNE clusters included 3 biologically-relevant clusters that were not found in any other embeddings.
The largest of these (cluster 3 with 289 samples) corresponded to a subclade of A2 that was previously associated with a reassortment to a different neuraminidase background \cite{Potter2019}.
Cluster 7 (N=22 samples) descended from a subclade of A1a and its samples carried substitutions at known epitope sites (HA1:140M and HA1:193S).
Cluster 8 (N=29 samples) descended from clade A1 and contained 14 samples with a substitution at another known epitope site (HA1:135K).
The 7 MDS clusters were nearly twice as far from expert clades as UMAP clusters (normalized VI=0.13), suggesting that MDS's ability to accurately represent genetic distance did not correspond to high-resolution clusters.
MDS clusters captured most of the larger clades (A1, A2, A3, A4, 3c2.A, and 3c3.A), but they also collected A1 and its descendants into the same cluster and suffered from more unassigned samples than the other embeddings.
The PCA embedding produced the lowest accuracy (normalized VI=0.18) and fewest clusters (N=3).
Despite the appearance of distinct clusters associated with clades in the PCA embedding (Fig~\ref{fig:seasonal-influenza-h3n2-ha-embeddings}), each component of the PCA embedding appeared to form only two clusters corresponding to some of the most distantly related and ancestral clades (3c2.A, 3c3.A, and A2).
These results indicate that nonlinear embeddings of t-SNE and UMAP could be better-suited for clustering and classification than linear embeddings from PCA and MDS.

\begin{figure}[!h]
\includegraphics[width=\columnwidth]{figures/flu-2016-2018-ha-embeddings-by-cluster.png}
\caption{{\bf Embeddings of H3N2 HA samples colored by HDBSCAN cluster.}
  Trees colored by cluster (left column) show where clusters map in relation to phylogenetic clades.
  Clusters from PCA, MDS, t-SNE, and UMAP (right column) generally recapitulate groups of viruses into clades without inferring ancestral relationships.}
\label{fig:seasonal-influenza-h3n2-ha-2016-2018-clusters}
\end{figure}

To understand whether these embedding methods and optimal cluster parameters could effectively cluster previously unseen sequences, we applied each method to the ``late'' H3N2 HA dataset (2018--2020), clustered sequences in the embedding space with HDBSCAN using the optimal minimum distance threshold from the ``early'' dataset, and calculated the accuracy of the cluster assignments based on previously defined clades.
Unlike the early H3N2 HA dataset, the late dataset represented less genetic diversity with most clades descending from clade A1b with at least one additional characteristic HA1 amino acid substitutions  and some older samples from clades A2 and 3c3.A (\nameref{S4_Fig_late_flu_embeddings_by_clade}).
Clusters from all four methods generally captured relevant phylogenetic clades (Fig.~\ref{fig:seasonal-influenza-h3n2-ha-2018-2020-clusters} and \nameref{S4_Fig_late_flu_embeddings_by_clade}).
The MDS clusters most accurately captured expert clades (normalized VI=0.06) with 8 clusters corresponding to clades 3c3.A, A2, A3, A1b/94N, A1b/135K, A1b/135N, A1b/137F, and A1b/131K merged with A1b/197R (Fig.~\ref{fig:seasonal-influenza-h3n2-ha-2018-2020-clusters} and \nameref{S5_Fig_late_flu_mds_embeddings}).
Similarly, MDS cluster 4 merged a separate subset of A1b/131K samples with their descendants in A1b/94N.
MDS failed to create a cluster for A1b/186D samples, leaving these all as unassigned.
Both t-SNE and UMAP followed closely in accuracy (normalized VI=0.09) with 6 and 5 clusters, respectively.
Both sets of clusters generally matched those from MDS except that the most recent clades clustered into broader groups with their ancestral clades (e.g., A1b/135K and A1b/131K).
PCA produced clusters with the lowest accuracy (normalized VI=0.11), but these 6 clusters were not qualitatively much different from t-SNE and UMAP clusters.
PCA clusters split clade 3c3.A into two groups and merged A1b/94N with a larger cluster of its ancestral clade, A1b/131K, and that ancestor's other descendants.

\begin{figure}[!h]
% TODO: remove includegraphics commands in final submission; figures must be uploaded separately from the manuscript.
\includegraphics[width=\columnwidth]{figures/flu-2018-2020-ha-embeddings-by-clade.png}
\caption*{{\bf S4 Fig. Phylogeny of late (2018--2020) influenza A/H3N2 HA sequences (top) and reduced dimensionality embeddings of genetic sequences into two dimensions by PCA (middle left), MDS (middle right), t-SNE (bottom left), and UMAP (bottom right).}}
\end{figure}

\begin{figure}[!h]
% TODO: remove includegraphics commands in final submission; figures must be uploaded separately from the manuscript.
\includegraphics[width=\columnwidth]{figures/flu-2018-2020-mds-by-clade.png}
\caption*{{\bf S5 Fig. MDS embeddings for late (2018--2020) influenza A/H3N2 HA sequences showing all three components.}}
\end{figure}

\begin{figure}[!h]
\includegraphics[width=\columnwidth]{figures/flu-2018-2020-ha-embeddings-by-cluster.png}
\caption{{\bf The embeddings colored by their HDBSCAN label, with the distance threshold defined by the threshold that preserved the greatest amount of clade relationships.}
The chart for PCA (top left), MDS (middle left), t-SNE (middle left), and UMAP (bottom left) generally recapitulate groups of viruses into clades without inferring ancestral relationships, and the trees on the righthand side describes how these clade grouping appear on the tree, which does infer ancestral relations.}
\label{fig:seasonal-influenza-h3n2-ha-2018-2020-clusters}
\end{figure}

\subsection*{Joint embeddings of hemagglutinin and neuraminidase genomes identify seasonal influenza virus A/H3N2 reassortment events}

\begin{figure}[!h]
\includegraphics[width=\columnwidth]{figures/flu-2016-2018-ha-na-embeddings-by-mcc.png}
\caption{{\bf Embeddings of seasonal influenza HA-only (first column) and concatenated HA/NA sequences (second column) colored by TreeKnit Maximally Compatible Clades (MCC) label.}
}
\label{fig:seasonal-influenza-h3n2-ha-na-2016-2018-embeddings}
\end{figure}

\subsection*{SARS-CoV-2 clusters recapitulate emerging lineage designations}

\begin{figure}[!h]
\includegraphics[width=\columnwidth]{figures/sarscov2-embeddings-by-Nextstrain_clade-clade.png}
\caption{{\bf Embeddings of SARS-CoV-2 sequences collected between January 1, 2020 and January 1, 2021 colored by Nextstrain clade label.}
}
\label{fig:sars-cov-2-2020-2021-embeddings}
\end{figure}

\begin{figure}[!h]
\includegraphics[width=\columnwidth]{figures/sarscov2-euclidean-distance-by-genetic-distance.png}
\caption{{\bf The mapping between Euclidean and genetic distance assess the strength of both the local and global structure of the embedding recapitulation.}
The scatterplot for PCA (upper left), MDS (upper right), t-SNE (lower left), and UMAP (lower right) consistently exhibit linear relationships for pairs of strains that differ by around 20 nucleotides.}
\label{fig:sars-cov-2-pairwise-distances}
\end{figure}

\begin{figure}[!h]
\includegraphics[width=\columnwidth]{figures/sarscov2-embeddings-by-cluster-vs-Nextstrain_clade.png}
\caption{{\bf Embeddings of SARS-CoV-2 sequences collected between January 1, 2020 and January 1, 2021 colored by embedding cluster and annotated by normalized VI to indicate accuracy of clusters for training data compared to expert clade assignment (Nextstrain clade).}
}
\label{fig:sars-cov-2-2020-2021-clusters-vs-Nextstrain-clade}
\end{figure}

\begin{figure}[!h]
\includegraphics[width=\columnwidth]{figures/sarscov2-test-embeddings-by-cluster-vs-Nextstrain_clade.png}
\caption{{\bf Embeddings of SARS-CoV-2 sequences collected between January 1, 2022 and July 5, 2023 colored by embedding cluster and annotated by normalized VI to indicate accuracy of clusters for training data compared to expert clade assignment (Nextstrain clade).}
}
\label{fig:sars-cov-2-2022-2023-clusters-vs-Nextstrain-clade}
\end{figure}

\section*{Discussion}

TBD.

\section*{Conclusion}

TBD.

\section*{Supporting information}

% Include only the SI item label in the paragraph heading. Use the \nameref{label} command to cite SI items in the text.
\paragraph*{S1 Fig.}
\label{S1_Fig_simulated_flu_errors}
{\bf Distribution of mean absolute errors (MAE) between observed and predicted pairwise genetic distances per embedding method parameters for simulated influenza-like populations.} Each panel shows boxplots of MAEs for a specific embedding method (PCA, MDS, t-SNE, and UMAP) and a given combination of method parameters. Boxplots reflect median, upper and lower quartiles, and the range of values.

\paragraph*{S2 Fig.}
\label{S2_Fig_simulated_coronavirus_errors}
{\bf Distribution of mean absolute errors (MAE) between observed and predicted pairwise genetic distances per embedding method parameters for simulated coronavirus-like populations.} Each panel shows boxplots of MAEs for a specific embedding method (PCA, MDS, t-SNE, and UMAP) and a given combination of method parameters. Boxplots reflect median, upper and lower quartiles, and the range of values.

\paragraph*{S3 Fig.}
\label{S3_Fig_simulated_representative_mds_embeddings}
{\bf Representative MDS embeddings for simulated populations using optimal parameters per pathogen (rows) and showing all three components.}

\paragraph*{S4 Fig.}
\label{S4_Fig_late_flu_embeddings_by_clade}
{\bf Phylogeny of late (2018--2020) influenza A/H3N2 HA sequences (top) and reduced dimensionality embeddings of genetic sequences into two dimensions by PCA (middle left), MDS (middle right), t-SNE (bottom left), and UMAP (bottom right).}

\paragraph*{S5 Fig.}
\label{S5_Fig_late_flu_mds_embeddings}
{\bf MDS embeddings for late (2018--2020) influenza A/H3N2 HA sequences showing all three components.}

\section*{Acknowledgments}
TBD.

\nolinenumbers

% Either type in your references using
% \begin{thebibliography}{}
% \bibitem{}
% Text
% \end{thebibliography}
%
% or
%
% Compile your BiBTeX database using our plos2015.bst
% style file and paste the contents of your .bbl file
% here. See http://journals.plos.org/plosone/s/latex for
% step-by-step instructions.
%

% TODO: copy/paste bbl file contents below instead of using standard bibliography commands.
\bibliographystyle{plos2015}
\bibliography{cartography}

% \begin{thebibliography}{10}

% \bibitem{bib1}
% Conant GC, Wolfe KH.
% \newblock {{T}urning a hobby into a job: how duplicated genes find new
%   functions}.
% \newblock Nat Rev Genet. 2008 Dec;9(12):938--950.

% \bibitem{bib2}
% Ohno S.
% \newblock Evolution by gene duplication.
% \newblock London: George Alien \& Unwin Ltd. Berlin, Heidelberg and New York:
%   Springer-Verlag.; 1970.

% \bibitem{bib3}
% Magwire MM, Bayer F, Webster CL, Cao C, Jiggins FM.
% \newblock {{S}uccessive increases in the resistance of {D}rosophila to viral
%   infection through a transposon insertion followed by a {D}uplication}.
% \newblock PLoS Genet. 2011 Oct;7(10):e1002337.

% \end{thebibliography}

\end{document}
